\begin{abstract}
This work is a final project for a Bachelor's degree in Computer Engineering, specializing in Computing.
The project is completed at the Universitat Politècnica de Catalunya (UPC), specifically at the Facultat d'Informàtica de Barcelona (FIB).
Gerard Escudero Bakx directs the project, with supervision from Edelmira Pasarella Sanchez.
The project builds upon the master's thesis completed by Royo-Sales et al. \cite{royo_sales_algorithm_2021} in 2021,supervised by Edelmira Pasarella Sanchez.
Royo-Sales et al. \cite{royo_sales_algorithm_2021} thesis focused on developing a Haskell library for the dynamic pipeline paradigm and its application to an algorithm for incremental enumerating bitriangles (IEBT).
This work involves understanding Royo-Sales et al. \cite{royo_sales_algorithm_2021} work, including the Haskell library, its potential application to other algorithms, and the implementation of the IEBT algorithm.
The project aims to provide a guide for utilizing the library in implementing any algorithm that leverages the dynamic pipeline paradigm.
Additionally, it proposes a set of improvements to the library itself.
Furthermore, the project explores potential enhancements to the IEBT algorithm, accompanied by performance tests to validate these improvements.
\\


Este trabajo es un proyecto final del Grado en Ingeniería Informática, con especialidad en Computación.
El proyecto se realiza en la Universitat Politècnica de Catalunya (UPC), específicamente en la Facultat d'Informàtica de Barcelona (FIB).
Gerard Escudero Bakx dirige el proyecto, con la supervisión de Edelmira Pasarella Sanchez.
El proyecto se basa en la tesis de máster realizada por Royo-Sales et al. \cite{royo_sales_algorithm_2021} en 2021, supervisada por Edelmira Pasarella Sanchez.
La tesis de Royo-Sales et al. \cite{royo_sales_algorithm_2021} se centró en desarrollar una librería de Haskell para el paradigma dynamic pipeline y su aplicación a un algoritmo para la enumeración incremental de bitriángulos (IEBT).
Este trabajo implica comprender el trabajo de Royo-Sales et al. \cite{royo_sales_algorithm_2021}, incluyendo la librería de Haskell, su potencial aplicación a otros algoritmos y la implementación del algoritmo IEBT.
El proyecto tiene como objetivo proporcionar una guía para utilizar la librería en la implementación de cualquier algoritmo que se base en el paradigma de canalizaciones dinámicas.
Además, propone un conjunto de mejoras a la propia librería.
Finalmente, el proyecto explora posibles mejoras del algoritmo IEBT, acompañadas de pruebas de rendimiento para validar dichas mejoras.
\\

Aquest treball és un projecte final per a un Grau en Enginyeria Informàtica, amb especialitat en Computació.
El projecte es realitza a la Universitat Politècnica de Catalunya (UPC), específicament a la Facultat d'Informàtica de Barcelona (FIB).
Gerard Escudero Bakx dirigeix el projecte, sota la supervisió d'Edelmira Pasarella Sanchez.
El projecte es basa en la tesi de màster realitzada per Royo-Sales et al. \cite{royo_sales_algorithm_2021} en 2021, supervisada per Edelmira Pasarella Sanchez.
La tesi de Royo-Sales et al. \cite{royo_sales_algorithm_2021} es va centrar en desenvolupar una biblioteca Haskell per al paradigma dynamic pipeline i la seva aplicació a un algorisme per a l'enumeració incremental de bitriangles (IEBT).
Aquest treball implica comprendre el treball de Royo-Sales et al. \cite{royo_sales_algorithm_2021}, incloent-hi la biblioteca Haskell, la seva potencial aplicació a altres algorismes i la implementació de l'algorisme IEBT.
El projecte té com a objectiu proporcionar una guia per a utilitzar la biblioteca en la implementació de qualsevol algorisme que es basi en el paradigma de canalitzacions dinàmiques.
A més, proposa un conjunt de millores a la pròpia biblioteca.
Finalment, el projecte explora possibles millores de l'algorisme IEBT, acompanyades de proves de rendiment per a validar aquestes millores.
\end{abstract}