\chapter{Improving the IEBT algotihm}
This is the second part of my work, where I will try to find and argue for different improvements for the implementation of the algorithm for incremental enumeration of Bitriagles (IEBT).
This implementation developed by Juan Pablo uses the Dynamic Pipeline library with which I have been working.
The entire first part of my work has been necessary to fully understand the operation of both the dynamic pipeline paradigm itself and the operation of the Haskell library.

\section{Improving th structures}
The first topic I wanted to address was the data types used to store the different sets.
The algorithm needs to store different vertex structures for later matching.
These are the different definitions found in the code:

\begin{figure}[ht]
    %\centering
    \begin{tabular}{c}
        \begin{lstlisting}[escapeinside={(*}{*)}]
type LowerVertex = Int
type UpperVertex = Int
type Edge = (UpperVertex, LowerVertex)

data Command = ByVertex IntSet
            | ByEdge (Set Edge)
            | Count
            | AllBT
            | NoCommand
            | End
        deriving (Show, Read)

data W = W
  { _wLowerVertex :: LowerVertex
  , _wWedges      :: IntSet
  }
  deriving Show

data Triplet = Triplet Int Int Int
data Pair = Pair Int Int

type UT = (IntSet, IntSet, IntSet)

data DW = DW
  { _dwLower :: Pair
  , _dwUpper :: UT
  }

newtype DWTT = DWTT [DW]
  deriving newtype (Semigroup, Monoid)

data BT = BT
  { _btLower :: Triplet
  , _btUpper :: UT
  }

newtype BTTT = BTTT [BT]
  deriving newtype (Semigroup, Monoid)

data BTResult = RBT Q (Int, Int, Int, Int, Int, Int, Int)
              | RC  Q Int

data FilterState = Adj W
                 | DoubleWedges DWTT
                 | BiTriangles BTTT
        \end{lstlisting}
    \end{tabular}
    \caption{Data type definitions of IEBT implementation}
    \label{fig:HC40}
\end{figure}


We can see how it represents vertices with integers Int and edges as tuples of integers (Int, Int).
The most important thing to analyze are the set representations, as their impact on the performance of the algorithm is crucial.
It uses the IntSet structure to represent sets of vertices and also a tuple of 3 IntSets (IntSet, IntSet, IntSet) to represent the set of upper vertex of the Aggregated Doble Wedge and Aggregated Bitriangles. \cite[][Page52]{}
Also taking into account the use of lists to represent the set of Aggregated Double Wedges and Aggregated Bitriangles. \\

We will be analizing structure by structure trying to find if the structure used is the most efficient.
\subsection{Set of vertices representation}
The IntSet structure belongs to the Haskell IntSet library \cite[][]{}, which represents an ordered set of Ints.
This structure is very efficient because it takes advantage of the range limit of Ints to limit many operations to a cost of O(min(n,W)), where n is the number of elements and W is the number of bits in the Int representation (32 or 64)
Additionally, this structure is based on big-endian patricia trees, which have very good performance for set intersection and union operations.

