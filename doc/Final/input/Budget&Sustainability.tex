\chapter{Budget and Sustainability}
\section{Budget}
In this section I will identify the costs of the project. 
I will break down the costs into different categories, then I will estimate the costs of each one and finally I will propose a management control system to monitor the costs.
\subsection{Identification of costs}
Here are all the cost group by categories that I will consider in this project.
\subsubsection*{Hardware}
In my project the only hardware that is needed is a computer.
We need a computer for all the process: research, programming, writing the documentation, etc.
If we needed to test our algorithm with a large dataset, we maybe would need a computer with a high performance or a more specialized hardware, but in this work I will not consider this case.
I will consider a generic computer / laptop.
\subsubsection*{Software}
In this project is only planned to use a few software tools.
Starting with the programming language, I will use Haskell.
Haskell has a compiler called Glasgow Haskell Compiler (GHC) and would be sufficient for the project.
I'm also using Visual Studio Code as a text editor, for both writing the code and the documentation.
The next software that I will use is \LaTeX, for writing the documentation.
I also will use GitHub for the version control of the project, that uses Git.
\subsubsection*{Human resources}
Truthy I'm the only human resource that is working in this project, but I will consider me as a person assuming different roles.
As mentinoded in the planning section, I will be taking 3 diferents roles:
\begin{itemize}
    \item Researcher: For tasks RL2, SA1 and some documentation
    \item Haskell Developer: For tasks RL3, all FA tasks and all SA tasks
    \item Project Manager: For all G and D tasks
\end{itemize}
\subsubsection*{Other costs}
I will not consider other costs as the cost of the internet connection, the electricity and the cost of the paper and ink for printing the documentation.
As I consider very specific costs and do not have a huge impact in the total cost.
Other cost that could be considered is the cost of the transportation, cost of the supervisor and co-supervisor work and other stuff that could be needed for the project.
As I consider that these costs are difficult to estimate, I will not consider them in this project.
\subsection{Cost estimates}
Now that all costs are identifies, here I will estimate the cost of each one, considering the duration of the project and the cost of each resource.
\subsubsection*{Hardware}
I will consider a standard price for a good laptop, that is around 1000 euros.
Obviously a laptop have a longer life than the project duration, so I will consider the cost of the laptop as a cost that is distributed in time, amortization.
$$
\text{Amortization} = \text{Laptop Cost} \cdot \frac{\text{Project Duration}}{\text{Laptop Lifespan}}
$$ 
Normaly, laptops have a lifespan of 3 to 5 years with a avegare of 40 hours per week, so I will consider 4 years, 40 hours per week as the lifespan of the laptop. %\cite{}
Also I will consider that the whole project will need the laptop, so there will be a total of 460 hours. 
$$
\text{Amortization} = 1000 \text{€} \cdot \frac{460 \text{hours}}{4 \text{years} \cdot 52 \frac{\text{weeks}}{\text{year}} \cdot 40 \frac{\text{hours}}{\text{week}}} = 55,29 \text{€}
$$
\subsubsection*{Software}
All the software that I will use is free \cite{}. So the total amount of money spend in software will be 0€.
\subsubsection*{Human resources}
I will be tacking the average salary of each role from glassdoor \cite{GlassDoorResearcher} \cite{GlassdoorProjectManager}\cite{GlassdoorSoftwareDeveloper}, a well known website for job search and salary information.
We will check Spain salaries and we will consider 2000 hours per year.
Here we can see the cost of the human resources, assuming a social security multiplier of 1.3.

\begin{table}[H]
    \begin{adjustwidth}{-1in}{-1in} % Adjust margins by -1 inch on both sides
    \centering
    \begin{tabular}{|c|c|c|c|c|}
    \hline
    \textbf{Role} & \textbf{Avg. Salary (€/h)} & \textbf{Hours} & \textbf{Total (€)} & \textbf{Total with Social Security (€)} \\ 
    \hline
    Researcher & 15 & 85 hours & 1275 & 1657.5\\
    \hline
    Haskell Developer & 18 & 200 hours & 3600 & 4680\\
    \hline
    Project Manager & 20 & 175 hours & 3500 & 4550\\
    \hline
    \hline
    \textbf{Total} & 18.2 & 460 & 8375 &  10887.5\\
    \hline
    \end{tabular}
    \caption{Human Resources Cost, self elaborated}
    \label{human_resources}
    \end{adjustwidth}
    \end{table}
\subsubsection*{Risk plan}
Some risks may appear during the project, so I will add some extra money to the budget to cover it.
\begin{enumerate}
    \item Laptop damage: The laptop could be damaged and need to be repaired. I will consider 100€ for this risk.
    \item Extra hours: The project could take more time than expected and each hour of the workers is so expensive. Considering 10 \% more time, gives us 837 €, so I will add 1000€ to the budget.
    \item For other risks and unexpected costs, I will add 10 \% of the total budget, to cover it, 1100 €.
\end{enumerate}
Puting all together, the total amount of money for the unexpected costs is 2037 €.
\subsubsection*{Total budget}
Lets sumarize all the costs and calculate the total budget.
\begin{table}[H]
    \begin{adjustwidth}{-1in}{-1in} % Adjust margins by -1 inch on both sides
    \centering
    \begin{tabular}{|c|c|}
    \hline
    \textbf{Source} & \textbf{Cost (€)} \\ 
    \hline
    Hardware &  55,29 \\
    \hline
    Software & 0 \\
    \hline
    Human Resources & 10887.5 \\
    \hline
    Unexpected & 2037 \\
    \hline
    \hline
    \textbf{Total} & \textbf{12979.79}  \\
    \hline
    \end{tabular}
    \caption{Total budget, self elaborated}
    \label{total_budget}
    \end{adjustwidth}
    \end{table}
\subsection{Management control}
Once we have our budget, I can create a indicator so I can monitor the deviation from the initial plan.
This indicator (C) will be the difference between the expected and actual cost:
$$
C = C_estimado - C_real
$$ 
I can use this indicator both in a general way and with each subsection of the budget, in order to be able to analyze what may be causing a variation in the budget and be able to fix it, either by modifying the budget itself or by restructuring the project.
\section{Sustainability report}
\subsection{Economic Dimension}
After analized my project and made the budget, we can observe that the main cost of the project is the human resources.
Asuming that my plannign is correct, I can confirm that no unnecessary costs are being generated, beacause I'm using free software and the minimun hardware required.
Also this project have the goal of improving an implementation of an algorithm, meaning a possible improvement in the efficiency of the algorithm, which could lead to a reduction in time, which means a reduction of the cost.
\subsection{Environmental Dimension}
As mentioned before, the only environmental impact of the project is the energy consumed by the laptop. 
The energy and impact are minimum, so we can consider this project as environmentally friendly.
Also as previous point, the project could lead to a reduction in time, which means a reduction of the energy consumed.
A good point about projects like mine is that their useful life can be considered 'unlimited'.
This is because it is a research and development project, providing utility at all times.
\subsection{Social Dimension}
In personal terms, I think it will have a great impact on my life and career as a student, since this project will be the first 'big project', which will help me learn from mistakes for future projects.
This project may not have a big social impact, but it can do your part in the world of computing and especially in the world of Haskell programming.
